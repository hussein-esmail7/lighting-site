% Streamyard Notes Final.tex
% Hussein Esmail
% Created 2021 03 20
% Updated 2021 03 20
% Description: [DESCRIPTION]

\documentclass{article}
\title{Streamyard Demo Notes}
\author{Hussein Esmail}
\date{October 18, 2020} % Current date rather than creation date

\usepackage{xcolor}     % Used for specific colours (must be defined before pagecolor!)
\usepackage{pagecolor}  % Used for setting page colour
\usepackage{amsmath}    % Used for aligning equations
\usepackage{hyperref}   % This block contains information used for PDF metadata
\usepackage{graphicx}   % Used to display images

\usepackage[none]{hyphenat} % Used to wrap words when a word finishes instead of in the middle of a word


\graphicspath{{./}}     % Images are in the same folder
\hypersetup{            % PDF metadata
    colorlinks=false,           % hyperlinks will be black
    linkbordercolor=blue,        % color of internal links
    linkcolor=black,
    citebordercolor=black,      % color of links to bibliography
    citecolor=black,
    filebordercolor=magenta,    % color of file links
    filecolor=black,
    urlbordercolor=cyan,         % color of external links
    urlcolor=black,
    pdfborder=false,  % underline links instead of boxes
    pdfborderstyle={/U/W 1},  % underline links instead of boxes
    pdfborder={0 0 0},
    pdftitle={Streamyard Notes Final}, 
    pdfauthor={Hussein Esmail}, 
    pdfsubject={Subject here}, % TODO 
    pdfkeywords={}
}

\newcommand{\comment}[1]{}  % \comment{Multiline comments can go here.}
\newcommand{\image}[1]{[\textbf{IMAGE MISSING #1}]} % #1 is whatever descriptor you want to display about the image

\pagecolor{white}       % default page colour (\newpagecolor{c} can set as well in document)
\color{black}           % default text colour

\begin{document} % Official beginning of the document. 
\maketitle
\newpage
\begingroup
\hypersetup{linkcolor=black}
\tableofcontents
\endgroup
\newpage

% BULLET LIST
% \begin{itemize}
%     \item Bulleted list
% \end{itemize}

% NUMBERED LIST
% \begin{enumerate}
%     \item Numbered list
% \end{enumerate}

% ALIGNED EQUATION (aligns at "&" symbol, \\ = new line)
% \begin{align}
%     1+3 &= 1+1+1 \\ &= 1+2 \\ &= 3
% \end{align}

%%%%%%%%%%%%%%%%%%%%%%%%
% Text here
%%%%%%%%%%%%%%%%%%%%%%%%

\section{Introduction}
This document acts as somewhat as a tutorial for the streaming software Streamyard. It can stream to YouTube and Facebook. It is assumed that the reader of this document is going to be the Streamyard host of a performance. During a show, it would be extremely hard to have a smooth show when the Streamyard host and the performance host are the same person. 

\subsection{Required Software}
To be on Streamyard (host or guest), you need one of these browsers: Google Chrome, Brave, Opera, Firefox.

\subsection{Getting Started}
When you go to \href{https://streamyard.com}{https://streamyard.com}, there isn’t a login and password. Instead, you type in the email of the account (in this case that is Brian Goldenberg’s email, \href{mailto:bgolden@yorku.ca}{bgolden@yorku.ca}), and a code is emailed to him. He then can give you that code so that you can log in. Though, if you don’t log out when you’re done it should still be logged in when you come back.

\section{Managing What the Audience Sees}
\subsection{First Glance}
When you open the correct broadcast, you will see a big blue rectangle in the top left corner. That is what the audience members see during the stream. Below that is the “backstage”, where guests remain before they are brought onstage by the host. If you bring multiple peoples’ video feeds onstage, you can use the row of video arrangement buttons between the stream screen and the backstage members.

To access other panels like the chat or branding options, there is a toolbar to the very right of the screen with the buttons, Comments, Banners, Brand, Private Chat, and Settings.

One thing to note about broadcasts, is that they will automatically go away after you’ve “gone live” and exited Streamyard, but this doesn’t delete anything because of Branding (see the Branding section for more info).

\subsection{Assets}
\subsubsection{``Banners''}
In the “Banners” tab, you can make as many banners as you want. You can only have one be onscreen at a time, and you can have it on for as long as you want (you have to manually turn it on and off). This is great for saying what the performance name is during the performance or who is performing. They can be rearranged in order and sorted into folders depending on what show it belongs to.

\subsubsection{``Branding''}
In the “Branding” tab, there are many options on what you can display. You can change the brand colour (by using a colour picker or inputting a hex code), add a logo to the top right corner of the screen, add an overlay to the entire stream feed, play video clips, and change the background image. For all of these asset types, you can add as many as you want.

For the logos and overlays, it is highly recommended that you have some transparency in these because it would look bad if there isn’t. Especially for the overlay, because otherwise that would be the only thing the viewer would see if it was on. 

Keep in mind the only supported video format for uploading video clips is .mp4. If it's not that, you will need to convert it somehow. When you play video clips, it starts playing on the stream right when you click it, and do not provide any warning for how much time is left. In previous shows, we would use videos as the intro and outro, and when a specific image was on, what would be our queue to go live and when to end the recording. Another issue with video clips, is that if there is a banner onscreen, that must be manually turned off if you want to play a video, or else it will still be on top of the video clip as well.

For the background images, it is recommended to have something dull like a solid colour so that it isn’t distracting from the stream itself. If there is no background image, it will be solid black, which doesn’t look too visually appealing.

\section{Managing Performers}
\subsection{Inviting People}
To invite people to the stream, at the bottom of the window beside the “Leave Studio” button, there is a button “Invite”. When you press this, a popup appears with a link. This link can be sent to anyone joining the stream. If people are prompted to sign in trough YouTube or Facebook, the best thing for them is to sign in through YouTube with their Passport York account through Google. When they join, they go to the backstage of the stream, and to be onscreen, they have to be added by the Streamyard host.

The best way for people in the show to communicate with the host is to use the private chat if you’re backstage, or to talk out loud if you’re on the stream.

\subsection{Backstage Limits}\label{sec:Limits}
The backstage of Streamyard has a limit of around 10 people, so to manage how many people are there, it is recommended to ask people to join at a certain time before their act, and kick the performer out when their act is finished (in case they forget to leave on their own). To kick a person out, first they can’t be on the main video stream. The next thing is to hover on their individual screen in the backstage section and press the 3 dots in the right corner. Then press “Kick from Stream” (not “Ban from Stream”). 

\subsection{Stream Comments}
When streaming to YouTube, stream comments on YouTube can show up on Streamyard, and you can also click comments to feature them on the stream video feed, by clicking them on then off. The issue is if you want to show multiple comments one after the other, you have to click each one on and off. 

\subsection{Video Feeds}

% If accessing from website: https://hussein-esmail7.github.io/site/guides/VCP/documents/VCP_Streamyard_Camera_views.jpg
\includegraphics{VCP_Streamyard_Camera_views} % VCP_Streamyard_Camera_views.jpg in the same folder

When people are onscreen, there are many options for how you want to display their feeds. You can try out each of these during the tech run, just note that the 2nd view does not look too good with 3 people because each person has to be in the middle third of their camera which might not look well. You can also drag screens around, and the viewer will not see anything until you let go of what you are dragging (they will not see the overlay animation that shows up for you). 

On each person’s video feed in the backstage section, each top left corner has a “Solo Layout” button. When this is turned on, that person’s feed will take up the whole stream screen. When it is turned off, the layout will revert to whatever it was before it was turned on. If Solo Layout is on for one person, then another person’s is turned on, their screen will replace the Streamyard feed.

While on the topic of peoples’ individual buttons, when a specific person is not on the stream, you can kick them out of backstage. Instructions on this is in the \textbf{\nameref{sec:Limits} (\ref{sec:Limits})} section on page \pageref{sec:Limits}.
\end{document} % Official end of the document. 
